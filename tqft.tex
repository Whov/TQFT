\documentclass[a4paper, 11pt]{article}
\usepackage[a4paper, left=3cm, bottom=4cm, right=3cm]{geometry}

\usepackage{beppe_package_eng} 
\usepackage{simplewick}
\usepackage{simpler-wick}

\renewcommand{\S}{\mathcal{S}}
\bibliographystyle{alpha}

\date{\today}
\author{Giuseppe Bogna, Bruno Bucciotti, Paolo Tognini}
\title{Quantum Fields and Topology}

\begin{document}
	\maketitle
	\tableofcontents
	\clearpage
	
	%%PARTE DI PAOLO
	\section{Introduzione}
	
	Partiamo con un breve ripasso di teorica 1. Molto del materiale della prima parte si può trovare nei primi capitoli di \cite{anselmi}, in cui vengono sviluppati parte dei metodi funzionali che tratteremo.
	
	\subsection{Interaction picture}
	Separiamo la lagrangiana in parte libera $\mathcal{L}_0$ al più quadratica nei campi e parte di interazione $\mathcal{L}_{int}$. Analogamente $\mathcal{H} = \mathcal{H}_0+\mathcal{H}_{int}$. Assumiamo temporaneamente per semplicità che $\mathcal{L}_{int}$ non dipenda dalle derivate dei campi, così $\mathcal{L}_{int} = -\mathcal{H}_{int}$. $\mathcal{H}_0$ definisce una evoluzione libera; passando in rappresentazione di interazione si ha $\tilde{H}_I(t) = e^{iH_0t}H_{int}e^{-iH_0t}$ e $S_I = e^{iH_0t}S_{int}e^{-iH_0t}$.
	
	\subsection{Path integral}
	
	
	\subsection{Funzione di partizione Z[j]}
	In questa sezione definiamo la $Z[J]$ e la calcoliamo nel caso della teoria libera. Mostriamo poi che la $Z[J]$ genera tutte le funzioni di correlazione a n punti. Dimostriamo che $Z[J] = e^{iS_I[-i\frac{\delta}{\delta J}]} Z_0[J]$ e osserviamo infine il collegamento fra l'espansione della $Z[J]$ e i diagrammi di Feynman.
	
	
	Definiamo Z[J] dove $J(x)$ è una funzione dello spaziotempo in $\R$ che assegnamo.
	
	\[ Z[J] = \bra{0}T\left[e^{iS_I[\hat{\phi}]}e^{i\int{\hat{\phi}(x)J(x)}\d^4x}\right]\ket{0} \]
	\[ = \int \mathcal{D}\phi e^{iS[\phi] + i \int \phi J \d^4 x} \]
	Notare che $Z[J]$ è il valore di aspettazione sul vuoto dell'operatore $e^{i\int{\hat{\phi}(x)J(x)}\d^4x}$. Il vuoto che compare nella formula è quello definito dalla teoria libera.
	\\
		
	Facciamo un esempio	in cui la teoria è libera ($S_I=0$)
	\[S_0 = \int \d^4 x (\frac{1}{2} \partial_\alpha \phi \partial^\alpha \phi - \frac{m^2}{2}\phi^2)\]
	\[Z_0[J]=\bra{0}Te^{i\int \d^4 x J(x)\phi(x)}\ket{0} = \]
	\[\bra{0}T\sum_{n=0}^{+\infty} \frac{i^n}{n!}\int \d^4x_1 \dots \d^4 x_n J(x_1)\dots J(x_n) \phi(x_1)\dots \phi(x_n)\ket{0}\]
	se $n$ è dispari ($n=2p+1$) il valore di aspettazione è nullo:
	\[\bra{0}T\phi(x_1)\dots \phi(x_{2p+1})\ket{0}\]
	Se $n$ è pari($n=2p$) abbiamo:
	\[\bra{0}T(\phi(x_1)\dots\phi(x_{2p})\ket{0} = \sum_{j=2}^{2p}\wick{\c\phi(x_1) \c\phi(x_j)} \dots \bra{0}T\hat {\phi(x_1)} \phi(x_2) \dots \hat{\phi(x_j)}\dots \phi(x_{2p})\ket{0}\]
	dove $\hat {\phi(x_j)}$ indica che $\phi(x_j)$ non c'è.\\
	Notiamo che la cosa può essere fatta $p$ volte. Ricorsivamente abbiamo
	\[ \sum_{p=0}^{+\infty} \frac{i^{2p}}{(2p)!}\int \d^4x_1 \dots \d^4 x_{2p} J(x_1)\dots J(x_{2p}) \left(\phi_1\dots\phi_{2p}\, \mathrm{contratti\, in\, tutti\, i\, modi\, possibili}\right)\]
	Per simmetria del resto dell'integrando sotto permutazione delle variabili $x_i$ di integrazione, la somma su tutte le contrazioni dei $\phi$ si riduce a
	\[ \sum_{p=0}^{+\infty} i^{2p}\frac{(2p-1)!!}{(2p)!}\int \d^4x_1 \dots \d^4 x_{2p} J(x_1)\dots J(x_{2p}) \wick{\c \phi_1 \c \phi_2} \dots \wick{\c \phi_{2p-1} \c \phi_{2p}} \]
	ricordando che $\wick{\c \phi_1 \c \phi_2} = i \Delta(x_1-x_2)$
	
	\[= \sum_{p=0}^{+\infty} \frac{(-i)^p}{p!2^p} \left( \int \d^4 x \d^4 y J(x)\Delta(x-y)J(y)\right)^p\]
	dove abbiamo usato che gli integrali si dividono in $p$ coppie tutte uguali, abbiamo raccolto le $i$ accanto ai $\Delta$ e abbiamo diviso i fattoriali: $\dfrac{(2p-1)!!}{(2p)!} = \dfrac{(2p-1)(2p-3)\dots}{(2p)(2p-1)\dots} = \dfrac{1}{2^p p!}$.

	Segue che
	\[Z_0[J] = e^{-\frac{i}{2}\int \d^4 x \d^4 y J(x)\Delta(x-y)J(y)} \]
	\\
	
	Trattiamo ora le funzioni di correlazione. Ricordando dall'appendice qualche nozione di derivazione funzionale si ha
	%TODO fare appendice con derivate funzionali!
	\[ \langle T\phi(x_1)\phi(x_2)\dots \phi(x_n)\rangle\ = \left. \frac{1}{Z[0]}{\frac{-i\delta}{\delta J(x_1)}\frac{-i\delta}{\delta J(x_2)}\dots \frac{-i\delta}{\delta J(x_n)}}Z[J] \right|_{J=0} \]
	dove si è diviso per $Z[0]$ per normalizzare la transizione vuoto-vuoto (tale fattore di normalizzazione è infinito per la maggior parte delle teorie). La dimostrazione funziona sia per la definizione di $Z[J]$ data con il path integral sia con quella canonica, si noti infatti che il $T$ ordinamento porta i campi nella posizione corretta.

	Se evitiamo di settare $J=0$ si ha più in generale che, essendo $F(\phi)$ un polinomio nei campi,
	\[ \int \mathcal{D}\phi F(\phi) e^{iS}e^{i\int J\phi \d^4x} = F[-i\frac{\delta}{\delta J}] \int \mathcal{D}\phi e^{iS}e^{i\int J \phi \d^4x} \]
	allora concludiamo come caso particolare che
	\[ Z[J] = \int \mathcal{D}\phi e^{iS_0}e^{i\int J\phi \d^4x} e^{iS_I[\phi]} = \sum_{n=0}^{\infty} \frac{1}{n!} \int \mathcal{D}\phi e^{iS_0}e^{i\int J\phi \d^4x} (iS_I(\phi))^n \]
	\[ = \sum_{n=0}^{\infty} \frac{1}{n!} \left(iS_I[-i\frac{\delta}{\delta J}]\right)^n \int \mathcal{D}\phi e^{iS_0}e^{i\int J\phi \d^4x} = 
	e^{iS_I[-i\frac{\delta}{\delta J}]} Z_0[J] \]
	cioè possiamo agire sulla $Z_0$ libera con un particolare operatore e ottenere la $Z[J]$ interagente.
	
	Accenniamo infine che espandendo $Z[J]$ si ottiene una serie in termini delle funzioni di correlazione. Facendo un esempio si ha che $\langle \phi(x_1) \phi(x_2)\rangle$ (il cosiddetto \emph{propagatore vestito}) è una serie di diagrammi dei quali il primo è il propagatore "nudo", i successivi sono correzioni ad esso. In generale $Z[J]$ è la somma di tutti i diagrammi di Feynman della teoria (anche disconnessi).
	
	\subsection{W[J]}
	Definiamo
	\[ Z[J] = e^{W{[J]}} \]
	e mostriamo che, se $Z[J]$ era la somma di tutti i diagrammi di Feynman, $W[J]$ è la somma dei soli diagrammi connessi.
	
	Iniziamo dal caso libero. Qui $Z_0[J] = e^{-\frac{i}{2}\int \d^4 x \d^4 y J(x)\Delta(x-y)J(y)}$, che espanso in serie è la serie in $n$ di tutti i diagrammi di Feynman della teoria. Poichè la teoria è libera non ci sono vertici interni e l'unica possibilità è raggruppare fra loro $n$ propagatori liberi. L'unico diagramma connesso è il termine dell'esponenziale con $n=1$, ovverosia l'esponente ($W_0[J]$).
	Dunque nel caso libero $W_0[J]$ è la somma di tutti i diagrammi connessi della teoria.
	
	Da qui è possibile seguire la dimostrazione in \cite{anselmi} (pag.34). Noi procediamo in modo diverso. La $Z[J]$ è la somma di tutti i diagrammi, divisi per il fattore di simmetria del diagramma; inoltre, trascurando i fattori di simmetria, dato un diagramma disconnesso il suo contributo è il prodotto dei contributi dei diagrammi che lo compongono (cioè gli integrali si fattorizzano in quelli delle componenti connesse).
	
	Supponiamo di fare un elenco dei diagrammi connessi. Un diagramma generico $D$ è descritto univocamente dalla sequenza $\{n_i\}$ di numeri che contano quante volte compare il diagramma connesso $i$esimo. Il fattore di simmetria $S(D)$ di $D$ sarà il prodotto di due contributi: una parte viene dal fattore di simmetria di ciascuna componente connessa $i$, e tale fattore è elevato alla $n_i$; poi c'è un fattore $n_i!$ che viene dalla possibilità di permutare i diagrammi connessi.
	\[ Z[J] = \sum \dfrac{D}{S(D)} = \sum_{\{n_i\}} \prod_i \left(\dfrac{d_i}{S(d_i)}\right)^{n_i}\frac{1}{n_i!} = \prod_i \sum_{n=0}^{\infty} \left( \dfrac{d_i}{S(d_i)} \right)^n \dfrac{1}{n!} = \prod_{i} e^{\dfrac{d_i}{S(d_i)}} = e^{\sum_i \dfrac{d_i}{S(d_i)}} \]
	\[ Z[J] = e^{W[J]} \]
	
	\subsection{U[$\phi$]}
	Dato $\Phi$ campo classico reale, definiamo
	\[ U[\Phi] = \bra{0}T\left[e^{iS_I[\hat{\phi}+\Phi]}\right]\ket{0} \]%FORSE = \int \mathcal{D}\phi e^{iS[\phi+\Phi]} \]
	Se poniamo che l'azione di $a$, $a^\dagger$ sia
	\[ a_\vec{k} = \int\d^4x \dfrac{e^{ikx}}{\sqrt{2E_\vec{k}}} \dfrac{\delta}{\delta \Phi(x)},\qquad
	a_\vec{k}^\dagger = \int\d^4x \dfrac{e^{-ikx}}{\sqrt{2E_\vec{k}}} \dfrac{\delta}{\delta \Phi(x)} \]
	allora si trova UNA COSA FALSA!%TODO: fixare! Non stai inserendo phi e basta, ci sono potenze a caso.
	
	Consideriamo l'azione interagente nella teoria $S = \int \d^4x \frac{1}{2}(\partial_\mu\phi\partial^\mu\phi-m^2)-\frac{g}{4!}\phi^4$ e determiniamo i diagrammi se aggiungiamo il campo $\Phi$ esterno
	\[ S_{int}[\phi+\Phi] = \int\d^4x -\frac{g}{4!} [\phi^4+4\phi^3\Phi+6\phi^2\Phi^2+4\phi\Phi^3+\Phi^4] \]
	%TODO: fare i diagrammini
	
	La struttura dei diagrammi è universale, al di là della divergenza dei loro contributi. Notiamo che la differenza principale con $Z[J]$ è che le gambe esterne per $Z$ sono quantistiche (con sorgente $J(x)$ classica), mentre per $U$ le gambe esterne sono puramente classiche: se amputassimo le parti classiche, i diagrammi per $U$ sarebbero senza gambe esterne.
	
	\subsection{$\Gamma[\phi]$}
	Iniziamo definendo
	\[ \Phi[J](x) = \dfrac{\delta W[J]}{\delta J(x)} \]
	Supponiamo di poter invertire questo funzionale (tipicamente questo è esplicitamente possibile solo in modo perturbativo) e definiamo $\hat{J}$ come il funzionale tale che
	\[ J(x) = \hat{J}[\Phi](x) \]
	
	
	\section{Rinormalizzazione}
	Prendiamo come esempio la teoria $\phi^4$ 
	\[ S = \int \d^4x \frac{1}{2}(\partial_\mu\phi\partial^\mu\phi-m^2)-\frac{g}{4!}\phi^4 \]
	e studiamo il contributo del diagramma a un loop
	%Diagramma
	\[ \int\d^4x \d^4y \d^4z [i\Delta(x-y) iJ(y)][i\Delta(x-z)iJ(z)][i\Delta(x-x)] \]
	Esplicitiamo le funzioni $\Delta$ chiamando $p,q$ l'impulso proveniente da $y,z$ e $k$ l'impulso nel loop.
	\[ = i^5 \int \d^4x\d^4y\d^4z \frac{\d^4p}{(2\pi)^4}\frac{\d^4q}{(2\pi)^4}\frac{\d^4k}{(2\pi)^4} \frac{e^{-ip(x-y)}}{p^2-m^2+i\epsilon}\frac{e^{-iq(x-z)}}{q^2-m^2+i\epsilon}\frac{1}{k^2-m^2+i\epsilon} J(y)J(z) \]
	L'integrale in $y,z$ è una trasformata di fourier, quello in $x$ dà una delta di $p+q$
	\[ = i^5 \int \frac{\d^4p}{(2\pi)^4} \frac{|\tilde{J}(p)|^2}{(p^2-m^2+i\epsilon)}\int\frac{\d^4k}{(2\pi)^4}\frac{1}{k^2-m^2+i\epsilon} \]
	Il primo termine non dà problemi, supponendo che $\tilde{J}$ decada a infinito, mentre il secondo contributo (proveniente unicamente dal loop) diverge come $\Lambda^2$ nel cutoff $\Lambda$.
	%SE qualcuno mi controlla il conto qui sopra please, ma dovrebbe andare.
	
	In modo analogo si osserva che il diagramma associato a $i^2\Delta^2(x-y)$ diverge come $\ln(\Lambda)$.%TODO: fare diagramma
	
	Osserviamo che la divergenza nel valore dell'integrale si traduce nello spazio delle posizioni in una \emph{ambiguità} nella definizione del prodotto di due distribuzioni, cioè non è definito $\Delta(x-y)\Delta(x-y)$.
	
	
	Abbiamo:
	\[Z[J] = \bra{0}T(e^{iS_I[\phi]}e^{i\int{\phi(x)J(x)}}\ket{0} = e^{iS_I[-i\frac{\delta}{\delta J}]} Z_0[J] = e^{iS_I[-i\frac{\delta}{\delta J}]} e^{-\frac{i}{2}J\Delta J}\]
	È vera inoltre
	\[U[\Phi]=\bra{0}Te^{i S_I [\phi+\Phi]}\ket{0} = \left. e^{iS_I[\Phi - i\frac{\delta}{\delta J}]Z_0[J]} \right|_{J=0}\]
	\[U[\Phi]= \left. e^{iS_I[\Phi-i\frac{\delta}{\delta J}]} e^{-\frac{i}{2}\int J\Delta J} \right|_{J=0}\]
	A questo punto invoca una arcana formula che ha appreso da Coleman. Sembra di essere in un corso di magia!\\
	Proposizione: Se $F(x_j)$ e $G(x_k)$, $x = x_i = \{x_1,x_2,\dots x_n\}$ allora
	\[\left. F(-i\frac{\d}{\d x_j})G(x_k) = G(-i\frac{\d}{\d y_k}) F(y_j) e^{ix\cdot y} \right|_{y=0}\]
	Dove $x\cdot y = x_1 y_1 + x_2 y_2 + \dots x_n y_n$, e chiaramente a destra la derivata agisce anche sull'esponenziale!\\
	Coleman dice, astutamente: vediamo se è vera per onde piane:
	\[F(x_j)=e^{ib_j x_j}\]
	\[G(x_j)=e^{ic_k x_k}\]
	È vera per onde piane. Ma tutto è fatto da onde piane! Fourier a me! E così, è fatta.\\
	Vediamo meglio:
	\[F(-i\frac{\delta}{\delta x_j})G(x_k) = e^{ib_j (-i\frac{\d}{\d x_J})} e^{ic_k x_k} = e^{b_j \frac{\d}{\d x^j}} e^{ic_k x_k} = e^{ic_k (x_k + b_k)}\]
	Se fate il conto anche sul lato destro, torna. È un conto banale. Mi rifiuto di farlo.\\
	\\
	\noindent Ritorniamo a noi. Perché ho invocato il potere degli antichi? Perché ho
	\[Z[J] = e^{iS_I[-i\frac{\d}{\d J}]}e^{-\frac{i}{2}\int J\Delta J}\]
	E voglio scambiare il coso con la derivata con il resto. Applicando l'antico incantesimo, trovo:
	\[Z[J] = \left. e^{-\frac{i}{2} \int \d^4x \d^4y (-i\frac{\delta}{\delta K(x)}) \Delta(x-y) (-i\frac{\delta}{\delta K(y)})} e^{iS_I[K(x)]}e^{i\int \d^4x K(x)J(x)}\right|_{y=0}\]
	Cerchiamo di sistemare il primo pezzo con la nota identità:
	\[e^{\frac{i}{2}\int \d^4x \d^4y \frac{\delta}{\delta K(x)} \Delta(x-y) \frac{\delta}{\delta K(y)}} e^{i\int \d^4 K(w)J(w)} = e^{i\int \d^4w J(w)K(w)} e^{-\frac{i}{2}\int \d^4x \d^4y J(x)\Delta(x-y)J(y)} e^{\frac{i}{2}\int \d^4x \d^4y \frac{\delta}{\delta K(x)} \Delta(x-y) \frac{\delta}{\delta K(y)}} e^{-\int \d^4x \d^4y \frac{\delta}{\delta K(x)} \Delta(x-y) J(y)}\]
	In notazione più compatta:
	\[e^{\frac{i}{2} \delta_K \Delta \delta_K} e^{iK \cdot J} = e^{iJ \cdot K}e^{-\frac{i}{2}J\Delta J}e^{\frac{i}{2}\delta_K \Delta \delta_K} e^{-\delta_K \Delta J}\]
	Ricordiamo che in MQ
	\[e^{p\Delta p}e^{iqb} = e^{iqb} (e^{-iqb}e^{p\Delta p}e^{iqb}) = e^{iqb} e^{(p-ib)\Delta(p+ib)} \]
	E si vede che è proprio la stessa cosa! (si vede, si vede...).
	\[Z[J]= e^{-\frac{i}{2}\int \d^4 x \d^4 y J(x) \Delta(x-y) J(y)} = \left. e^{\frac{i}{2}\int \d^4 x \d^4 y \frac{\delta}{\delta K(x)} \Delta(x-y) \frac{\delta}{\delta K(y)}}e^{iS_I[K-\Delta J]} \right|_{K=0}\]
	dove ricordiamo che $J-\Delta J$ è uguale a
	\[J(x) - \int \Delta(x-y)J(y)\]
	Questa notazione è così ovvia che non serve parlarne.
	Ricordiamo piuttosto che con
	\[U(\Phi) = e^{\frac{i}{2}\int \d^4x \d^4y \frac{\delta}{\delta \Phi(x)}\Delta(x-y)\frac{\delta}{\delta \Phi(y)}} e^{iS_I[\Phi]}\]
	posso compattare ancora di più la formula precedente, ottenendo:
	\[Z[J] = Z_0[J] U[-\Delta J]\]
	Ricordiamo inoltre:\\
	QFT a livello perturbative = Contrazione di Wick con diagrammi di Feynman + Rinormalizzazione che fissa tutte le ambiguità\\
	(MI SEMBRA TENDENZIALMENTE FALSO...).\\
	
	
	
	\noindent Don't worry be happy!\\
	\[a^\dagger \ket{:(} = \ket{:)}\]
	
	%%FINE PARTE DI PAOLO
	
	\newpage
	\bibliography{bibliografia}
\end{document}